% \begin{center}
     % \Large{\textbf{Java Cheat Sheet}} \\
% \end{center}

\section{Pure Python}
\subsection{Types \& Transform}

\begin{tabular}{@{}l|l@{}}
    \multirow{3}{*}{Basic Types}  & {8.3e5, int(x[,base]), long(x[,base]), hex(), oct()} \\
    {}                            & {'word'} \\
    {}                            & {1.5+0.5j, complex(real[,imag])} \\
    \hline
    \multirow{2}{*}{To Number}    & {float(x)} \\
    {}                            & {ord(c)\{char to integer\}} \\
    \hline
    {To String}                   & {chr(x), unichar(x), str(x), repr(x)\{Object\}} \\
    \hline
    {Object}                      & {tuple(s), list(s), eval(str)\{run\}} \\
    \hline
    {Type}                        & {type(x), type('') is str} \\
    \hline
    {unchangable}                 & {string, tuples,numbers} \\
    {changable}                   & {list, dict} \\
    \hline
    \multirow{2}{*}{common utils} & {type(),dir(),getattr(),hasattr(),isinstance()}\\
    {}                            & {id()}\\
    % \verb!letter!              & Letter (?). \\
    % \verb!latex!             & Large sans-serif font.
\end{tabular}
    % \item{eval(str)              # 用来计算在字符串中的有效Python表达式,并返回一个对象}
    % \item{tuple(s)               # 将序列 s 转换为一个元组}
    % \item{list(s)                # 将序列 s 转换为一个列表}

\subsection{String}

\command{str.strip().lstrip().rstrip(\',\')}{}
\command{cmp(sStr1,sStr2)                }{compare}
\command{str.index(str2)                 }{Search str2, Exception when failed}
\command{sStr1.find(sStr2)               }{-1: Not found}
\command{len(str)                        }{}
\command{S.lower()                       }{}
\command{S.upper()                       }{}
\command{S.swapcase()                    }{}
\command{S.capitalize()                  }{Upper the first char}
\command{[::-1]                          }{reverse}

\subsection{list}

\command{b = list(range(5))                                      }{Initialize from iteratable }
\command{ret=[0] * (length+1)                                    }{}
\command{c=[nu**2 for nu in b]                                   }{ list comprehension}
\command{d=[nu**2 for nu in b if nu < 3]                         }{ conditioned list comprehension}
\command{h=['re', 'bl'] + ['gr']                                 }{ list concatenation}
\command{}{}
\command{sorted([3, 2, 1])                                       }{}
\command{list.index(var)                                         }{ Search var, exception when failed}
\command{list.append(var)                                        }{}
\command{list.pop(idx)                                           }{}
\command{list.insert(idx,var)                                    }{}
\command{list.remove(var)                                        }{ remove the first var}
\command{list.count(var)                                         }{ count number for a var}
\command{list.extend(list)                                       }{ merge two lists}
\command{list.sort()                                             }{}
\command{list.reverse()                                          }{}
\command{a[start:to]                                             }{}
% multilist = [[0 for col in range(5)] for row in range(3)] }&{ 二维数组
\command{del L[1]                                                }{ Remove with index}
\command{del L[1:3]                                              }{ Remove with index range}

\subsection{Set}
\command{x = set('spam')     }{ y = set(['s','p','a','m'])}
\command{s = set([3,5,9,10]) }{}
\command{t = set("Hello")    }{}
\command{t.add(item)         }{}
\command{t.remove(item)      }{Exception when not exist}
\command{t.discard(item)     }{No exception}
\command{t.pop()             }{pop random on, exception when null}
\command{s.clear()           }{ }
\command{t.update([])        }{add multi itemi}
\command{$a = t|s$           }{t + s}
\command{$b = t\&s$          }{common for t \& s}
% {$c = t–s$           }&{difference}\\
% {$d = t^s$           }&{Those exists only once}\\

\subsection{Dictionary}

\command{a = \{'red': 'rouge', 'blue': 'bleu'\}}{}
\command{c = [value for key, value in a.items()]}{loop through contents}
\command{for k in dict:  for (k, v) in dict.items():}{}
\command{for k, v in dict.iteritems():}{}
\command{d = a.get('yellow', 'no translation found')}{}
\command{a.setdefault('extra', []).append('cyan')}{init key with default}
\command{del(dict["a"]}{}
\command{dict.pop("b")}{}
\command{if "key1" in D:}{}

\subsection{Sort}
\begin{lstlisting}[language=Python]
def cmp(e0, e1):
    if e0<e1:
        return 1
    elif e0>e1:
        return -1
    else:
        return 0
data = [1, 3, -1, 100, -50]
data.sort(cmp)
# or
sorted(data, cmp=None, key=lambda x: x.modified, reverse=False)
# or
import operator
x = {1: 2, 3: 4, 4: 3, 2: 1, 0: 0}
sorted_x = sorted(x.items(), key=operator.itemgetter(1))
\end{lstlisting}

\subsection{heapq}
\command{heapq.heapify(list)}{Make a heap}
\command{htapq.heappush(h, item)}{Add an item}
\command{htapq.heappop(h)}{Pop an item}
\command{htapq.heappushpop(h, item)}{Push then pop}
\command{htapq.heapreplace(h, item)}{Pop then push}
\command{htapq.nlargest(n, heap)}{}
\command{htapq.nsmallest(n, heap)}{}
\command{htapq->(key, data)}{Use key as comparator}

\subsection{Math}

\command{3 // 2}{integer division}
\command{3 ** 2}{exponent}
\command{sys.maxint}{max int in py2.0}
\command{sys.maxsize}{max int in py3.0}
\command{max(a, b)}{}
\command{min(a, b)}{}
\command{math.sqrt}{}

\subsection{Class}

\subsubsection{Functions}

\lstset{language=Python}
\begin{lstlisting}[language=Python]
class Point(object):
    def __init__(self, x):
        self.x = x

    def __del__(self):
    pass

    def __call__(self):
        print(self.x)

        #Object print
    def __repr__(self):
        return '{}: {} {}'.format(self.__class__.__name__,self.name,self.number)

        #Comparator
    def __cmp__(self, other):
        if hasattr(other, 'getKey'):
            return self.getKey().__cmp__(other.getKey())
\end{lstlisting}

\subsubsection{Decorator}
\begin{lstlisting}[language=Python]
# Decorator can be used to modify
# the behaviour of a function
class myDecorator(object):
    def __init__(self, f):
        self.f = f
    def __call__(self):
        print("call")
        self.f()

@myDecorator
def my_funct():
    print('func')
\end{lstlisting}


\subsubsection{Exception}
\begin{lstlisting}[language=Python]
try:
    open('/path/to/does/not/exist', 'rb')
# except (SystemExit, KeyboardInterrupt):
    # raise Exception
except Exception, e:
    logger.error('Failed to open file', exc_info=True)
\end{lstlisting}

\subsubsection{With}
\begin{lstlisting}[language=Python]
# auto close when finish following indent
with open(filename, 'a') as f:
    f.write("")

    '''
class OpenContext(object):
    def __init__(self, filename, mode):
    self.fp=open(filename, mode);

    def __enter(self):
    return self.fp

    def __exit__(self, exec_type, exc_val, exc_tb):
    self.fp.close();

    with OpenConetxt(filename) as f:
    f.write('hello');
    '''
\end{lstlisting}

% To make this come out properly in landscape mode, do one of the following
% 1.
%  pdflatex latexsheet.tex
%
% 2.
%  latex latexsheet.tex
%  dvips -P pdf  -t landscape latexsheet.dvi
%  ps2pdf latexsheet.ps


% If you're reading this, be prepared for confusion.  Making this was
% a learning experience for me, and it shows.  Much of the placement
% was hacked in; if you make it better, let me know...


% 2008-04
% Changed page margin code to use the geometry package. Also added code for
% \verb!\documentclass{!\textit{class}\verb!}!.  Use
% \verb!\begin{document}! to start contents and \verb!\end{document}! to
% end the document.


% \subsection{Common \texttt{documentclass} options}
% \newlength{\MyLen}
% \settowidth{\MyLen}{\texttt{letterpaper}/\texttt{a4paper} \ }
% \begin{tabular}{@{}p{\the\MyLen}%
                % @{}p{\linewidth-\the\MyLen}@{}}
% \texttt{10pt}/\texttt{11pt}/\texttt{12pt} & Font size. \\
% \texttt{letterpaper}/\texttt{a4paper} & Paper size. \\
% \texttt{twocolumn} & Use two columns. \\
% \texttt{twoside}   & Set margins for two-sided. \\
% \texttt{landscape} & Landscape orientation.  Must use
                     % \texttt{dvips -t landscape}. \\
% \texttt{draft}     & Double-space lines.
% \end{tabular}

% Usage:
% \verb!\documentclass[!\textit{opt,opt}\verb!]{!\textit{class}\verb!}!.


% \subsection{Packages}
% \settowidth{\MyLen}{\texttt{multicol} }
% \begin{tabular}{@{}p{\the\MyLen}%
                % @{}p{\linewidth-\the\MyLen}@{}}
% %\begin{tabular}{@{}ll@{}}
% \texttt{fullpage}  &  Use 1 inch margins. \\
% \texttt{anysize}   &  Set margins: \verb!\marginsize{!\textit{l}%
                        % \verb!}{!\textit{r}\verb!}{!\textit{t}%
                        % \verb!}{!\textit{b}\verb!}!.            \\
% \texttt{multicol}  &  Use $n$ columns:
                        % \verb!\begin{multicols}{!$n$\verb!}!.   \\
% \texttt{latexsym}  &  Use \LaTeX\ symbol font. \\
% \texttt{graphicx}  &  Show image:
                        % \verb!\includegraphics[width=!%
                        % \textit{x}\verb!]{!%
                        % \textit{file}\verb!}!. \\
% \texttt{url}       & Insert URL: \verb!\url{!%
                        % \textit{http://\ldots}%
                        % \verb!}!.
% \end{tabular}

% Use before \verb!\begin{document}!.
% Usage: \verb!\usepackage{!\textit{package}\verb!}!


% \subsection{Title}
% \settowidth{\MyLen}{\texttt{.author.text.} }
% \begin{tabular}{@{}p{\the\MyLen}%
                % @{}p{\linewidth-\the\MyLen}@{}}
% \verb!\author{!\textit{text}\verb!}! & Author of document. \\
% \verb!\title{!\textit{text}\verb!}!  & Title of document. \\
% \verb!\date{!\textit{text}\verb!}!   & Date. \\
% \end{tabular}

% These commands go before \verb!\begin{document}!.  The declaration
% \verb!\maketitle! goes at the top of the document.

% \subsection{Miscellaneous}
% \settowidth{\MyLen}{\texttt{.pagestyle.empty.} }
% \begin{tabular}{@{}p{\the\MyLen}%
                % @{}p{\linewidth-\the\MyLen}@{}}
% \verb!\pagestyle{empty}!     &   Empty header, footer
                                 % and no page numbers. \\
% \verb!\tableofcontents!      &   Add a table of contents here. \\

% \end{tabular}



% \section{Document structure}
% \begin{multicols}{2}
% \verb!\part{!\textit{title}\verb!}!  \\
% \verb!\chapter{!\textit{title}\verb!}!  \\
% \verb!\section{!\textit{title}\verb!}!  \\
% \verb!\subsection{!\textit{title}\verb!}!  \\
% \verb!\subsubsection{!\textit{title}\verb!}!  \\
% \verb!\paragraph{!\textit{title}\verb!}!  \\
% \verb!\subparagraph{!\textit{title}\verb!}!
% \end{multicols}
% {\raggedright
% Use \verb!\setcounter{secnumdepth}{!$x$\verb!}! suppresses heading
% numbers of depth $>x$, where \verb!chapter! has depth 0.
% Use a \texttt{*}, as in \verb!\section*{!\textit{title}\verb!}!,
% to not number a particular item---these items will also not appear
% in the table of contents.
% }

% \subsection{Text environments}
% \settowidth{\MyLen}{\texttt{.begin.quotation.}}
% \begin{tabular}{@{}p{\the\MyLen}%
                % @{}p{\linewidth-\the\MyLen}@{}}
% \verb!\begin{comment}!    &  Comment (not printed). Requires \texttt{verbatim} package. \\
% \verb!\begin{quote}!      &  Indented quotation block. \\
% \verb!\begin{quotation}!  &  Like \texttt{quote} with indented paragraphs. \\
% \verb!\begin{verse}!      &  Quotation block for verse.
% \end{tabular}

% \subsection{Lists}
% \settowidth{\MyLen}{\texttt{.begin.description.}}
% \begin{tabular}{@{}p{\the\MyLen}%
                % @{}p{\linewidth-\the\MyLen}@{}}
% \verb!\begin{enumerate}!        &  Numbered list. \\
% \verb!\begin{itemize}!          &  Bulleted list. \\
% \verb!\begin{description}!      &  Description list. \\
% \verb!\item! \textit{text}      &  Add an item. \\
% \verb!\item[!\textit{x}\verb!]! \textit{text}
                                % &  Use \textit{x} instead of normal
                        % bullet or number.  Required for descriptions. \\
% \end{tabular}




% \subsection{References}
% \settowidth{\MyLen}{\texttt{.pageref.marker..}}
% \begin{tabular}{@{}p{\the\MyLen}%
                % @{}p{\linewidth-\the\MyLen}@{}}
% \verb!\label{!\textit{marker}\verb!}!   &  Set a marker for cross-reference,
                          % often of the form \verb!\label{sec:item}!. \\
% \verb!\ref{!\textit{marker}\verb!}!   &  Give section/body number of marker. \\
% \verb!\pageref{!\textit{marker}\verb!}! &  Give page number of marker. \\
% \verb!\footnote{!\textit{text}\verb!}!  &  Print footnote at bottom of page. \\
% \end{tabular}


% \subsection{Floating bodies}
% \settowidth{\MyLen}{\texttt{.begin.equation..place.}}
% \begin{tabular}{@{}p{\the\MyLen}%
                % @{}p{\linewidth-\the\MyLen}@{}}
% \verb!\begin{table}[!\textit{place}\verb!]!     &  Add numbered table. \\
% \verb!\begin{figure}[!\textit{place}\verb!]!    &  Add numbered figure. \\
% \verb!\begin{equation}[!\textit{place}\verb!]!  &  Add numbered equation. \\
% \verb!\caption{!\textit{text}\verb!}!           &  Caption for the body. \\
% \end{tabular}

% The \textit{place} is a list valid placements for the body.  \texttt{t}=top,
% \texttt{h}=here, \texttt{b}=bottom, \texttt{p}=separate page, \texttt{!}=place even if ugly.  Captions and label markers should be within the environment.

% %---------------------------------------------------------------------------

% \section{Text properties}

% \subsection{Font face}
% \newcommand{\FontCmd}[3]{\PBS\verb!\#1{!\textit{text}\verb!}!  \> %
                         % \verb!{\#2 !\textit{text}\verb!}! \> %
                         % \#1{#3}}
% \begin{tabular}{@{}l@{}l@{}l@{}}
% \textit{Command} & \textit{Declaration} & \textit{Effect} \\
% \verb!\textrm{!\textit{text}\verb!}!                    & %
        % \verb!{\rmfamily !\textit{text}\verb!}!               & %
        % \textrm{Roman family} \\
% \verb!\textsf{!\textit{text}\verb!}!                    & %
        % \verb!{\sffamily !\textit{text}\verb!}!               & %
        % \textsf{Sans serif family} \\
% \verb!\texttt{!\textit{text}\verb!}!                    & %
        % \verb!{\ttfamily !\textit{text}\verb!}!               & %
        % \texttt{Typewriter family} \\
% \verb!\textmd{!\textit{text}\verb!}!                    & %
        % \verb!{\mdseries !\textit{text}\verb!}!               & %
        % \textmd{Medium series} \\
% \verb!\textbf{!\textit{text}\verb!}!                    & %
        % \verb!{\bfseries !\textit{text}\verb!}!               & %
        % \textbf{Bold series} \\
% \verb!\textup{!\textit{text}\verb!}!                    & %
        % \verb!{\upshape !\textit{text}\verb!}!               & %
        % \textup{Upright shape} \\
% \verb!\textit{!\textit{text}\verb!}!                    & %
        % \verb!{\itshape !\textit{text}\verb!}!               & %
        % \textit{Italic shape} \\
% \verb!\textsl{!\textit{text}\verb!}!                    & %
        % \verb!{\slshape !\textit{text}\verb!}!               & %
        % \textsl{Slanted shape} \\
% \verb!\textsc{!\textit{text}\verb!}!                    & %
        % \verb!{\scshape !\textit{text}\verb!}!               & %
        % \textsc{Small Caps shape} \\
% \verb!\emph{!\textit{text}\verb!}!                      & %
        % \verb!{\em !\textit{text}\verb!}!               & %
        % \emph{Emphasized} \\
% \verb!\textnormal{!\textit{text}\verb!}!                & %
        % \verb!{\normalfont !\textit{text}\verb!}!       & %
        % \textnormal{Document font} \\
% \verb!\underline{!\textit{text}\verb!}!                 & %
                                                        % & %
        % \underline{Underline}
% \end{tabular}

% The command (t\textit{tt}t) form handles spacing better than the
% declaration (t{\itshape tt}t) form.

% \subsection{Font size}
% \setlength{\columnsep}{14pt} % Need to move columns apart a little
% \begin{multicols}{2}
% \begin{tabbing}
% \verb!\footnotesize!          \= \kill
% \verb!\tiny!                  \>  \tiny{tiny} \\
% \verb!\scriptsize!            \>  \scriptsize{scriptsize} \\
% \verb!\footnotesize!          \>  \footnotesize{footnotesize} \\
% \verb!\small!                 \>  \small{small} \\
% \verb!\normalsize!            \>  \normalsize{normalsize} \\
% \verb!\large!                 \>  \large{large} \\
% \verb!\Large!                 \=  \Large{Large} \\  % Tab hack for new column
% \verb!\LARGE!                 \>  \LARGE{LARGE} \\
% \verb!\huge!                  \>  \huge{huge} \\
% \verb!\Huge!                  \>  \Huge{Huge}
% \end{tabbing}
% \end{multicols}
% \setlength{\columnsep}{1pt} % Set column separation back

% These are declarations and should be used in the form
% \verb!{\small! \ldots\verb!}!, or without braces to affect the entire
% document.


% \subsection{Verbatim text}

% \settowidth{\MyLen}{\texttt{.begin.verbatim..} }
% \begin{tabular}{@{}p{\the\MyLen}%
                % @{}p{\linewidth-\the\MyLen}@{}}
% \verb@\begin{verbatim}@ & Verbatim environment. \\
% \verb@\begin{verbatim*}@ & Spaces are shown as \verb*@ @. \\
% \verb@\verb!text!@ & Text between the delimiting characters (in this case %
                      % `\texttt{!}') is verbatim.
% \end{tabular}


% \subsection{Justification}
% \begin{tabular}{@{}ll@{}}
% \textit{Environment}  &  \textit{Declaration}  \\
% \verb!\begin{center}!      & \verb!\centering!  \\
% \verb!\begin{flushleft}!  & \verb!\raggedright! \\
% \verb!\begin{flushright}! & \verb!\raggedleft!  \\
% \end{tabular}

% \subsection{Miscellaneous}
% \verb!\linespread{!$x$\verb!}! changes the line spacing by the
% multiplier $x$.





% \section{Text-mode symbols}

% \subsection{Symbols}
% \begin{tabular}{@{}l@{\hspace{1em}}l@{\hspace{2em}}l@{\hspace{1em}}l@{\hspace{2em}}l@{\hspace{1em}}l@{\hspace{2em}}l@{\hspace{1em}}l@{}}
% \&              &  \verb!\&! &
% \_              &  \verb!\_! &
% \ldots          &  \verb!\ldots! &
% \textbullet     &  \verb!\textbullet! \\
% \$              &  \verb!\$! &
% \^{}            &  \verb!\^{}! &
% \textbar        &  \verb!\textbar! &
% \textbackslash  &  \verb!\textbackslash! \\
% \%              &  \verb!\%! &
% \~{}            &  \verb!\~{}! &
% \#              &  \verb!\#! &
% \S              &  \verb!\S! \\
% \end{tabular}

% \subsection{Accents}
% \begin{tabular}{@{}l@{\ }l|l@{\ }l|l@{\ }l|l@{\ }l|l@{\ }l@{}}
% \`o   & \verb!\`o! &
% \'o   & \verb!\'o! &
% \^o   & \verb!\^o! &
% \~o   & \verb!\~o! &
% \=o   & \verb!\=o! \\
% \.o   & \verb!\.o! &
% \"o   & \verb!\"o! &
% \c o  & \verb!\c o! &
% \v o  & \verb!\v o! &
% \H o  & \verb!\H o! \\
% \c c  & \verb!\c c! &
% \d o  & \verb!\d o! &
% \b o  & \verb!\b o! &
% \t oo & \verb!\t oo! &
% \oe   & \verb!\oe! \\
% \OE   & \verb!\OE! &
% \ae   & \verb!\ae! &
% \AE   & \verb!\AE! &
% \aa   & \verb!\aa! &
% \AA   & \verb!\AA! \\
% \o    & \verb!\o! &
% \O    & \verb!\O! &
% \l    & \verb!\l! &
% \L    & \verb!\L! &
% \i    & \verb!\i! \\
% \j    & \verb!\j! &
% !`    & \verb!~`! &
% ?`    & \verb!?`! &
% \end{tabular}


% \subsection{Delimiters}
% \begin{tabular}{@{}l@{\ }ll@{\ }ll@{\ }ll@{\ }ll@{\ }ll@{\ }l@{}}
% `       & \verb!`!  &
% ``      & \verb!``! &
% \{      & \verb!\{! &
% \lbrack & \verb![! &
% (       & \verb!(! &
% \textless  &  \verb!\textless! \\
% '       & \verb!'!  &
% ''      & \verb!''! &
% \}      & \verb!\}! &
% \rbrack & \verb!]! &
% )       & \verb!)! &
% \textgreater  &  \verb!\textgreater! \\
% \end{tabular}

% \subsection{Dashes}
% \begin{tabular}{@{}llll@{}}
% \textit{Name} & \textit{Source} & \textit{Example} & \textit{Usage} \\
% hyphen  & \verb!-!   & X-ray          & In words. \\
% en-dash & \verb!--!  & 1--5           & Between numbers. \\
% em-dash & \verb!---! & Yes---or no?    & Punctuation.
% \end{tabular}


% \subsection{Line and page breaks}
% \settowidth{\MyLen}{\texttt{.pagebreak} }
% \begin{tabular}{@{}p{\the\MyLen}%
                % @{}p{\linewidth-\the\MyLen}@{}}
% \verb!\\!          &  Begin new line without new paragraph.  \\
% \verb!\\*!         &  Prohibit pagebreak after linebreak. \\
% \verb!\kill!       &  Don't print current line. \\
% \verb!\pagebreak!  &  Start new page. \\
% \verb!\noindent!   &  Do not indent current line.
% \end{tabular}


% \subsection{Miscellaneous}
% \settowidth{\MyLen}{\texttt{.rule.w..h.} }
% \begin{tabular}{@{}p{\the\MyLen}%
                % @{}p{\linewidth-\the\MyLen}@{}}
% \verb!\today!  &  \today. \\
% \verb!$\sim$!  &  Prints $\sim$ instead of \verb!\~{}!, which makes \~{}. \\
% \verb!~!       &  Space, disallow linebreak (\verb!W.J.~Clinton!). \\
% \verb!\@.!     &  Indicate that the \verb!.! ends a sentence when following
                        % an uppercase letter. \\
% \verb!\hspace{!$l$\verb!}! & Horizontal space of length $l$
                                % (Ex: $l=\mathtt{20pt}$). \\
% \verb!\vspace{!$l$\verb!}! & Vertical space of length $l$. \\
% \verb!\rule{!$w$\verb!}{!$h$\verb!}! & Line of width $w$ and height $h$. \\
% \end{tabular}



% \section{Tabular environments}

% \subsection{\texttt{tabbing} environment}
% \begin{tabular}{@{}l@{\hspace{1.5ex}}l@{\hspace{10ex}}l@{\hspace{1.5ex}}l@{}}
% \verb!\=!  &   Set tab stop. &
% \verb!\>!  &   Go to tab stop.
% \end{tabular}

% Tab stops can be set on ``invisible'' lines with \verb!\kill!
% at the end of the line.  Normally \verb!\\! is used to separate lines.


% \subsection{\texttt{tabular} environment}
% \verb!\begin{array}[!\textit{pos}\verb!]{!\textit{cols}\verb!}!   \\
% \verb!\begin{tabular}[!\textit{pos}\verb!]{!\textit{cols}\verb!}! \\
% \verb!\begin{tabular*}{!\textit{width}\verb!}[!\textit{pos}\verb!]{!\textit{cols}\verb!}!


% \subsubsection{\texttt{tabular} column specification}
% \settowidth{\MyLen}{\texttt{p}\{\textit{width}\} \ }
% \begin{tabular}{@{}p{\the\MyLen}@{}p{\linewidth-\the\MyLen}@{}}
% \texttt{l}    &   Left-justified column.  \\
% \texttt{c}    &   Centered column.  \\
% \texttt{r}    &   Right-justified column. \\
% \verb!p{!\textit{width}\verb!}!  &  Same as %
                              % \verb!\parbox[t]{!\textit{width}\verb!}!. \\
% \verb!@{!\textit{decl}\verb!}!   &  Insert \textit{decl} instead of
                                    % inter-column space. \\
% \verb!|!      &   Inserts a vertical line between columns.
% \end{tabular}


% \subsubsection{\texttt{tabular} elements}
% \settowidth{\MyLen}{\texttt{.cline.x-y..}}
% \begin{tabular}{@{}p{\the\MyLen}@{}p{\linewidth-\the\MyLen}@{}}
% \verb!\hline!           &  Horizontal line between rows.  \\
% \verb!\cline{!$x$\verb!-!$y$\verb!}!  &
                        % Horizontal line across columns $x$ through $y$. \\
% \verb!\multicolumn{!\textit{n}\verb!}{!\textit{cols}\verb!}{!\textit{text}\verb!}! \\
        % &  A cell that spans \textit{n} columns, with \textit{cols} column specification.
% \end{tabular}

% \section{Math mode}
% For inline math, use \verb!\(...\)! or \verb!$...$!.
% For displayed math, use \verb!\[...\]! or \verb!\begin{equation}!.

% \begin{tabular}{@{}l@{\hspace{1em}}l@{\hspace{2em}}l@{\hspace{1em}}l@{}}
% Superscript$^{x}$       &
% \verb!^{x}!             &
% Subscript$_{x}$         &
% \verb!_{x}!             \\
% $\frac{x}{y}$           &
% \verb!\frac{x}{y}!      &
% $\sum_{k=1}^n$          &
% \verb!\sum_{k=1}^n!     \\
% $\sqrt[n]{x}$           &
% \verb!\sqrt[n]{x}!      &
% $\prod_{k=1}^n$         &
% \verb!\prod_{k=1}^n!    \\
% \end{tabular}

% \subsection{Math-mode symbols}

% % The ordering of these symbols is slightly odd.  This is because I had to put all the
% % long pieces of text in the same column (the right) for it all to fit properly.
% % Otherwise, it wouldn't be possible to fit four columns of symbols here.

% \begin{tabular}{@{}l@{\hspace{1ex}}l@{\hspace{1em}}l@{\hspace{1ex}}l@{\hspace{1em}}l@{\hspace{1ex}} l@{\hspace{1em}}l@{\hspace{1ex}}l@{}}
% $\leq$          &  \verb!\leq!  &
% $\geq$          &  \verb!\geq!  &
% $\neq$          &  \verb!\neq!  &
% $\approx$       &  \verb!\approx!  \\
% $\times$        &  \verb!\times!  &
% $\div$          &  \verb!\div!  &
% $\pm$           & \verb!\pm!  &
% $\cdot$         &  \verb!\cdot!  \\
% $^{\circ}$      & \verb!^{\circ}! &
% $\circ$         &  \verb!\circ!  &
% $\prime$        & \verb!\prime!  &
% $\cdots$        &  \verb!\cdots!  \\
% $\infty$        & \verb!\infty!  &
% $\neg$          & \verb!\neg!  &
% $\wedge$        & \verb!\wedge!  &
% $\vee$          & \verb!\vee!  \\
% $\supset$       & \verb!\supset!  &
% $\forall$       & \verb!\forall!  &
% $\in$           & \verb!\in!  &
% $\rightarrow$   &  \verb!\rightarrow! \\
% $\subset$       & \verb!\subset!  &
% $\exists$       & \verb!\exists!  &
% $\notin$        & \verb!\notin!  &
% $\Rightarrow$   &  \verb!\Rightarrow! \\
% $\cup$          & \verb!\cup!  &
% $\cap$          & \verb!\cap!  &
% $\mid$          & \verb!\mid!  &
% $\Leftrightarrow$   &  \verb!\Leftrightarrow! \\
% $\dot a$        & \verb!\dot a!  &
% $\hat a$        & \verb!\hat a!  &
% $\bar a$        & \verb!\bar a!  &
% $\tilde a$      & \verb!\tilde a!  \\

% $\alpha$        &  \verb!\alpha!  &
% $\beta$         &  \verb!\beta!  &
% $\gamma$        &  \verb!\gamma!  &
% $\delta$        &  \verb!\delta!  \\
% $\epsilon$      &  \verb!\epsilon!  &
% $\zeta$         &  \verb!\zeta!  &
% $\eta$          &  \verb!\eta!  &
% $\varepsilon$   &  \verb!\varepsilon!  \\
% $\theta$        &  \verb!\theta!  &
% $\iota$         &  \verb!\iota!  &
% $\kappa$        &  \verb!\kappa!  &
% $\vartheta$     &  \verb!\vartheta!  \\
% $\lambda$       &  \verb!\lambda!  &
% $\mu$           &  \verb!\mu!  &
% $\nu$           &  \verb!\nu!  &
% $\xi$           &  \verb!\xi!  \\
% $\pi$           &  \verb!\pi!  &
% $\rho$          &  \verb!\rho!  &
% $\sigma$        &  \verb!\sigma!  &
% $\tau$          &  \verb!\tau!  \\
% $\upsilon$      &  \verb!\upsilon!  &
% $\phi$          &  \verb!\phi!  &
% $\chi$          &  \verb!\chi!  &
% $\psi$          &  \verb!\psi!  \\
% $\omega$        &  \verb!\omega!  &
% $\Gamma$        &  \verb!\Gamma!  &
% $\Delta$        &  \verb!\Delta!  &
% $\Theta$        &  \verb!\Theta!  \\
% $\Lambda$       &  \verb!\Lambda!  &
% $\Xi$           &  \verb!\Xi!  &
% $\Pi$           &  \verb!\Pi!  &
% $\Sigma$        &  \verb!\Sigma!  \\
% $\Upsilon$      &  \verb!\Upsilon!  &
% $\Phi$          &  \verb!\Phi!  &
% $\Psi$          &  \verb!\Psi!  &
% $\Omega$        &  \verb!\Omega!
% \end{tabular}
% \footnotesize

% %\subsection{Special symbols}
% %\begin{tabular}{@{}ll@{}}
% %$^{\circ}$  &  \verb!^{\circ}! Ex: $22^{\circ}\mathrm{C}$: \verb!$22^{\circ}\mathrm{C}$!.
% %\end{tabular}

% \section{Bibliography and citations}
% When using \BibTeX, you need to run \texttt{latex}, \texttt{bibtex},
% and \texttt{latex} twice more to resolve dependencies.

% \subsection{Citation types}
% \settowidth{\MyLen}{\texttt{.shortciteN.key..}}
% \begin{tabular}{@{}p{\the\MyLen}@{}p{\linewidth-\the\MyLen}@{}}
% \verb!\cite{!\textit{key}\verb!}!       &
        % Full author list and year. (Watson and Crick 1953) \\
% \verb!\citeA{!\textit{key}\verb!}!      &
        % Full author list. (Watson and Crick) \\
% \verb!\citeN{!\textit{key}\verb!}!      &
        % Full author list and year. Watson and Crick (1953) \\
% \verb!\shortcite{!\textit{key}\verb!}!  &
        % Abbreviated author list and year. ? \\
% \verb!\shortciteA{!\textit{key}\verb!}! &
        % Abbreviated author list. ? \\
% \verb!\shortciteN{!\textit{key}\verb!}! &
        % Abbreviated author list and year. ? \\
% \verb!\citeyear{!\textit{key}\verb!}!   &
        % Cite year only. (1953) \\
% \end{tabular}

% All the above have an \texttt{NP} variant without parentheses;
% Ex. \verb!\citeNP!.


% \subsection{\BibTeX\ entry types}
% \settowidth{\MyLen}{\texttt{.mastersthesis.}}
% \begin{tabular}{@{}p{\the\MyLen}@{}p{\linewidth-\the\MyLen}@{}}
% \verb!@article!         &  Journal or magazine article. \\
% \verb!@book!            &  Book with publisher. \\
% \verb!@booklet!         &  Book without publisher. \\
% \verb!@conference!      &  Article in conference proceedings. \\
% \verb!@inbook!          &  A part of a book and/or range of pages. \\
% \verb!@incollection!    &  A part of book with its own title. \\
% %\verb!@manual!          &  Technical documentation. \\
% %\verb!@mastersthesis!   &  Master's thesis. \\
% \verb!@misc!            &  If nothing else fits. \\
% \verb!@phdthesis!       &  PhD. thesis. \\
% \verb!@proceedings!     &  Proceedings of a conference. \\
% \verb!@techreport!      &  Tech report, usually numbered in series. \\
% \verb!@unpublished!     &  Unpublished. \\
% \end{tabular}

% \subsection{\BibTeX\ fields}
% \settowidth{\MyLen}{\texttt{organization.}}
% \begin{tabular}{@{}p{\the\MyLen}@{}p{\linewidth-\the\MyLen}@{}}
% \verb!address!         &  Address of publisher.  Not necessary for major
                                % publishers.  \\
% \verb!author!           &  Names of authors, of format .... \\
% \verb!booktitle!        &  Title of book when part of it is cited. \\
% \verb!chapter!          &  Chapter or section number. \\
% \verb!edition!          &  Edition of a book. \\
% \verb!editor!           &  Names of editors. \\
% \verb!institution!      &  Sponsoring institution of tech.\ report. \\
% \verb!journal!          &  Journal name. \\
% \verb!key!              &  Used for cross ref.\ when no author. \\
% \verb!month!            &  Month published. Use 3-letter abbreviation. \\
% \verb!note!             &  Any additional information. \\
% \verb!number!           &  Number of journal or magazine. \\
% \verb!organization!     &  Organization that sponsors a conference. \\
% \verb!pages!            &  Page range (\verb!2,6,9--12!). \\
% \verb!publisher!        &  Publisher's name. \\
% \verb!school!           &  Name of school (for thesis). \\
% \verb!series!           &  Name of series of books. \\
% \verb!title!            &  Title of work. \\
% \verb!type!             &  Type of tech.\ report, ex. ``Research Note''. \\
% \verb!volume!           &  Volume of a journal or book. \\
% \verb!year!             &  Year of publication. \\
% \end{tabular}
% Not all fields need to be filled.  See example below.

% \subsection{Common \BibTeX\ style files}
% \begin{tabular}{@{}l@{\hspace{1em}}l@{\hspace{3em}}l@{\hspace{1em}}l@{}}
% \verb!abbrv!    &  Standard &
% \verb!abstract! &  \texttt{alpha} with abstract \\
% \verb!alpha!    &  Standard &
% \verb!apa!      &  APA \\
% \verb!plain!    &  Standard &
% \verb!unsrt!    &  Unsorted \\
% \end{tabular}

% The \LaTeX\ document should have the following two lines just before
% \verb!\end{document}!, where \verb!bibfile.bib! is the name of the
% \BibTeX\ file.
% \begin{verbatim}
% \bibliographystyle{plain}
% \bibliography{bibfile}
% \end{verbatim}

% \subsection{\BibTeX\ example}
% The \BibTeX\ database goes in a file called
% \textit{file}\texttt{.bib}, which is processed with \verb!bibtex file!.
% \begin{verbatim}
% @String{N = {Na\-ture}}
% @Article{WC:1953,
  % author  = {James Watson and Francis Crick},
  % title   = {A structure for Deoxyribose Nucleic Acid},
  % journal = N,
  % volume  = {171},
  % pages   = {737},
  % year    = 1953
% }
% \end{verbatim}


% \section{Sample \LaTeX\ document}
% \begin{verbatim}
% \documentclass[11pt]{article}
% \usepackage{fullpage}
% \title{Template}
% \author{Name}
% \begin{document}
% \maketitle

% \section{section}
% \subsection*{subsection without number}
% text \textbf{bold text} text. Some math: $2+2=5$
% \subsection{subsection}
% text \emph{emphasized text} text. \cite{WC:1953}
% discovered the structure of DNA.


% \rule{0.3\linewidth}{0.25pt}
% \scriptsize

% Copyright \copyright\ 2014 Winston Chang

% \href{http://wch.github.io/latexsheet/}{http://wch.github.io/latexsheet/}

